\documentclass{article}
\usepackage{hyperref}
\usepackage{listings}
\usepackage{color}
\usepackage{geometry}
\usepackage{graphicx}
\geometry{margin=1in}

\definecolor{codegray}{rgb}{0.5,0.5,0.5}
\definecolor{codepurple}{rgb}{0.58,0,0.82}

\lstdefinestyle{bash}{
  backgroundcolor=\color{white},
  basicstyle=\ttfamily\small,
  commentstyle=\color{codegray},
  keywordstyle=\color{blue},
  stringstyle=\color{codepurple},
  frame=single,
  breaklines=true,
  breakatwhitespace=true,
  showstringspaces=false,
  tabsize=2,
}

\begin{document}

\title{Perforce Tutorial}
\author{}
\date{}

\maketitle

\tableofcontents

\section{Introduction to Perforce}

Perforce Helix Core is a centralized version control system (VCS) that manages large codebases and digital assets. It is designed to handle complex development environments with speed and efficiency, making it a popular choice for large organizations and game development studios.

\subsection*{Key Benefits}

\begin{itemize}
    \item \textbf{Performance:} Optimized for large files and large numbers of files.
    \item \textbf{Security:} Granular access controls and audit trails.
    \item \textbf{Scalability:} Handles thousands of users and terabytes of data.
\end{itemize}

\section{Installation and Setup}

\subsection{Download and Install Perforce}

\begin{enumerate}
    \item \textbf{Download the Perforce Command-Line Client (P4):}
    \begin{itemize}
        \item Visit the \href{https://www.perforce.com/downloads}{Perforce Downloads} page.
        \item Choose the appropriate client for your operating system.
    \end{itemize}
    \item \textbf{Install the Client:}
    \begin{itemize}
        \item Follow the installation instructions specific to your OS.
    \end{itemize}
\end{enumerate}

\subsection{Set Up Environment Variables}

\begin{itemize}
    \item \textbf{P4PORT:} Specifies the address of the Perforce server.
    
    \begin{lstlisting}[style=bash]
export P4PORT=perforce:1666
    \end{lstlisting}
    
    \item \textbf{P4USER:} Your Perforce username.
    
    \begin{lstlisting}[style=bash]
export P4USER=your_username
    \end{lstlisting}
    
    \item \textbf{P4CLIENT:} Your workspace/client name.
    
    \begin{lstlisting}[style=bash]
export P4CLIENT=your_workspace
    \end{lstlisting}
\end{itemize}

\subsection{Authenticate with the Server}

\begin{lstlisting}[style=bash]
p4 login
\end{lstlisting}

\section{Perforce Concepts in Depth}

\subsection{Depot}

The \textbf{depot} is the central repository on the Perforce server where all versioned files are stored. It can contain multiple projects and branches.

\subsection{Workspace (Client)}

A \textbf{workspace} (also known as a client) is a local copy of files from the depot. It maps files in the depot to your local file system.

\subsubsection*{Creating a Workspace}

\begin{lstlisting}[style=bash]
p4 client
\end{lstlisting}

This command opens a text editor where you can define the workspace view and other settings.

\subsection{Changelists}

A \textbf{changelist} is a collection of file revisions that are submitted to the depot as a single unit. Each changelist has a unique number and a description.

\subsubsection*{Types of Changelists}

\begin{itemize}
    \item \textbf{Default Changelist:} The changelist where files are opened by default.
    \item \textbf{Numbered Changelist:} A user-created changelist with a specific number.
\end{itemize}

\subsection{Revision}

A \textbf{revision} is a specific version of a file in the depot. Revisions are numbered sequentially (e.g., \texttt{filename\#1}, \texttt{filename\#2}).

\subsection{Integration}

Integration in Perforce refers to the process of propagating changes between different branches. It encompasses both branching and merging.

\section{Basic Workflow with Examples}

Let's walk through a typical Perforce workflow with practical examples.

\subsection{Setting Up Your Workspace}

\textbf{Example:} Creating a new workspace named \texttt{alice\_workspace}.

\begin{lstlisting}[style=bash]
p4 client alice_workspace
\end{lstlisting}

In the editor that opens:

\begin{itemize}
    \item \textbf{Client:} \texttt{alice\_workspace}
    \item \textbf{Root:} \texttt{/Users/alice/projects/perforce}
    \item \textbf{View:}

\begin{verbatim}
//depot/... //alice_workspace/...
\end{verbatim}
\end{itemize}

Save and close the editor.

\subsection{Syncing Files from the Depot}

\textbf{Example:} Sync all files in the workspace to the latest revision.

\begin{lstlisting}[style=bash]
p4 sync
\end{lstlisting}

\textbf{Output:}

\begin{verbatim}
//depot/project/file1.txt#3 - added as /Users/alice/projects/perforce/project/file1.txt
//depot/project/file2.txt#5 - updated /Users/alice/projects/perforce/project/file2.txt
\end{verbatim}

\subsection{Editing Files}

Before editing a file, you need to open it for edit.

\textbf{Example:} Editing \texttt{file1.txt}.

\begin{lstlisting}[style=bash]
p4 edit project/file1.txt
\end{lstlisting}

\textbf{Output:}

\begin{verbatim}
//depot/project/file1.txt#3 - opened for edit
\end{verbatim}

Make your changes to \texttt{project/file1.txt} using your preferred text editor.

\subsection{Adding New Files}

\textbf{Example:} Adding a new file \texttt{file3.txt}.

\begin{enumerate}
    \item Create the file locally.

\begin{lstlisting}[style=bash]
echo "New content" > project/file3.txt
\end{lstlisting}

    \item Add the file to Perforce.

\begin{lstlisting}[style=bash]
p4 add project/file3.txt
\end{lstlisting}

    \textbf{Output:}

\begin{verbatim}
//depot/project/file3.txt#1 - opened for add
\end{verbatim}
\end{enumerate}

\subsection{Deleting Files}

\textbf{Example:} Deleting \texttt{file2.txt}.

\begin{lstlisting}[style=bash]
p4 delete project/file2.txt
\end{lstlisting}

\textbf{Output:}

\begin{verbatim}
//depot/project/file2.txt#5 - opened for delete
\end{verbatim}

\subsection{Reverting Changes}

\textbf{Example:} Reverting changes to \texttt{file1.txt}.

\begin{lstlisting}[style=bash]
p4 revert project/file1.txt
\end{lstlisting}

\textbf{Output:}

\begin{verbatim}
//depot/project/file1.txt#3 - was edit, reverted
\end{verbatim}

\subsection{Resolving Conflicts}

Conflicts occur when multiple users edit the same file concurrently.

\textbf{Example:} Resolving a conflict in \texttt{file1.txt}.

\begin{enumerate}
    \item Sync the latest changes.

\begin{lstlisting}[style=bash]
p4 sync
\end{lstlisting}

    \textbf{Output:}

\begin{verbatim}
//depot/project/file1.txt#4 - merging //depot/project/file1.txt
\end{verbatim}

    \item Resolve the conflict.

\begin{lstlisting}[style=bash]
p4 resolve
\end{lstlisting}

    \textbf{Options:}
    \begin{itemize}
        \item \texttt{e}: Launches an editor for manual merging.
        \item \texttt{accept yours/ theirs/ merge/ edit}
    \end{itemize}
\end{enumerate}

\subsection{Submitting Changes}

\textbf{Example:} Submitting changes in the default changelist.

\begin{lstlisting}[style=bash]
p4 submit
\end{lstlisting}

An editor opens to enter a changelist description.

\begin{itemize}
    \item \textbf{Change:} \texttt{new}
    \item \textbf{Description:} \texttt{Fixed bug in file1.txt and added file3.txt}
\end{itemize}

Save and close the editor.

\textbf{Output:}

\begin{verbatim}
Change 102 submitted.
\end{verbatim}

\section{Branching and Merging}

Branching allows you to diverge from the main line of development to work on features or fixes independently.

\subsection{Creating a Branch}

\textbf{Example:} Branching \texttt{//depot/project/...} to \texttt{//depot/branches/feature1/...}.

\begin{enumerate}
    \item Use the \texttt{integrate} command.

\begin{lstlisting}[style=bash]
p4 integrate //depot/project/... //depot/branches/feature1/...
\end{lstlisting}

    \textbf{Output:}

\begin{verbatim}
//depot/branches/feature1/file1.txt#1 - branch from //depot/project/file1.txt#3
\end{verbatim}

    \item Submit the branch.

\begin{lstlisting}[style=bash]
p4 submit -d "Created feature1 branch"
\end{lstlisting}
\end{enumerate}

\subsection{Making Changes in the Branch}

Edit files in the branch as usual.

\begin{lstlisting}[style=bash]
p4 edit //depot/branches/feature1/file1.txt
\end{lstlisting}

\subsection{Merging Changes Back to Main}

\textbf{Example:} Integrate changes from \texttt{feature1} back to \texttt{project}.

\begin{enumerate}
    \item Integrate changes.

\begin{lstlisting}[style=bash]
p4 integrate //depot/branches/feature1/... //depot/project/...
\end{lstlisting}

    \item Resolve conflicts if any.

\begin{lstlisting}[style=bash]
p4 resolve
\end{lstlisting}

    \item Submit the merged changes.

\begin{lstlisting}[style=bash]
p4 submit -d "Merged feature1 into main project"
\end{lstlisting}
\end{enumerate}

\section{Advanced Features}

\subsection{Shelving}

Shelving allows you to store your changes on the server without committing them to the depot.

\textbf{Example:} Shelving changes in a changelist.

\begin{lstlisting}[style=bash]
p4 shelve -c 103
\end{lstlisting}

\subsection{Labeling}

Labels are used to tag a group of file revisions for later reference.

\textbf{Example:} Creating a label for a release.

\begin{enumerate}
    \item Create the label specification.

\begin{lstlisting}[style=bash]
p4 label RELEASE_1.0
\end{lstlisting}

    \item In the editor, define the label view and description.
    \item Tag files with the label.

\begin{lstlisting}[style=bash]
p4 labelsync -l RELEASE_1.0 //depot/project/...
\end{lstlisting}
\end{enumerate}

\subsection{Permissions and Protections}

Manage user access with the \texttt{p4 protect} command (requires admin privileges).

\section{Conclusion}

By now, you should have a solid understanding of Perforce's basic and advanced features. Remember to practice these commands in a test environment to become more comfortable with Perforce operations.

\subsection*{Further Resources}

\begin{itemize}
    \item \textbf{Perforce Official Documentation:} \href{https://www.perforce.com/manuals}{https://www.perforce.com/manuals}
    \item \textbf{Perforce Visual Client (P4V):} A GUI tool for interacting with Perforce.
\end{itemize}

\end{document}

